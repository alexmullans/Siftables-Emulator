\title{Singularity Software\\Milestone 5}
\date{\today}

\documentclass[12pt]{article}
\usepackage[a4paper]{geometry}
\usepackage{makeidx}
\usepackage[acronym]{glossaries}
\usepackage{lscape}
\usepackage{amsmath}
\usepackage{graphicx}
\usepackage[final]{pdfpages}
% \usepackage{hyperref} % Makes links from ToC

\geometry{top=1.0in, bottom=1.0in, left=1.0in, right=1.0in} % Sets the margins

\setlength{\parindent}{0pt} % Fixes the paragraph spacing problem

% This is all for formatting and making the Table of Contents according to 
% spec. Don't play with it.
\makeatletter
\renewcommand\l@section[2]{%
  \ifnum \c@tocdepth >\z@
    \addpenalty\@secpenalty
    \addvspace{1.0em \@plus\p@}%
    \setlength\@tempdima{1.5em}%
    \begingroup
      \parindent \z@ \rightskip \@pnumwidth
      \parfillskip -\@pnumwidth
      \leavevmode \bfseries
      \advance\leftskip\@tempdima
      \hskip -\leftskip
      #1\nobreak\ 
      \leaders\hbox{$\m@th\mkern \@dotsep mu\hbox{.}\mkern \@dotsep mu$}
     \hfil \nobreak\hb@xt@\@pnumwidth{\hss #2}\par
    \endgroup
  \fi}
\makeatother

\makeindex

% Construct the glossary here
% Use the template below, then where the word appears (in the case below, computer), replace computer with \gls{computer}
\makeglossaries

\newglossaryentry{Sifteo Cubes}
{
  name={Sifteo Cubes},
  description={are small machines capable of loading programs and interacting with one another as well as responding to predefined movements}
}

\newglossaryentry{Object-Oriented Programming}
{
  name={Object-Oriented Programming},
  description={is a programming paradigm using objects to design applications}
}

\newglossaryentry{Windows}
{
  name={Windows},
  description={is a series of operating systems developed by Microsoft}
}

\newglossaryentry{Mac}
{
  name={Mac},
  description={is a series of lines of personal computers developed by Apple}
}

\newglossaryentry{Linux}
{
  name={Linux},
  description={is a Unix-based operating system based on free and open source software}
}

\newglossaryentry{cross-platform support}
{
  name={cross-platform support},
  description={is an attribute given to software implemented and operable on multiple computer platforms}
}

\newacronym{API}{API}{\glsadd{API}{Application Programming Interface}}

\newglossaryentry{APIg}
{
  name={Application Programming Interface},
  description={is an interface implemented by a software program that enables it to interact with other software}
}


\newglossaryentry{open source}
{
  name={open source},
  description={is an attribute given to software for which the source code is freely available}
}

\newacronym{IDE}{IDE}{\glsadd{IDE}{Integrated Development Environment}}

\newacronym{SDK}{SDK}{\glsadd{SDK}{Software Development Kit}}

\newglossaryentry{SDKa}
{
  name={Software Development Kit},
  description={is a collection of tools designed to help build software for a particular platform. It may include an \index{API} and an emulator of the target platform among other things.}
}

\newglossaryentry{IDEa}
{
  name={Integrated Development Environment},
  description={is software that provides a comprehensive work environment for computer programmers and software developers}
}


\newacronym{GUI}{GUI}{\glsadd{GUI}{Graphical User Interface}}

\newglossaryentry{GUIa}
{
  name={Graphical User Interface},
  description={is a visual way of allowing the user to interace with a computer program}
}

\newglossaryentry{version control}
{
  name={version control},
  description={is the management of documents and programs for a project over many versions in a well-organized manner}
}

\newglossaryentry{issue tracking system}
{
  name={issue tracking system},
  description={is a piece of software used to maintain a list of issues as generated during a project}
}

\newglossaryentry{usability study}
{
  name={usability study},
  description={is a manner of evaluating the design and user experience of a product by testing it on users}
}

\renewcommand*\arraystretch{1.5}

\begin{document}
\vspace*{\fill}
        \begin{center}
                \LARGE{Singularity Software} \\
                \LARGE{\textit{Milestone 5}} \\
                \vspace{.15in}
                \large{\today} \\
                \vspace{4in}
                By signing below, I approve the contents of the following document. \\
                \begin{table}[h]
                        \begin{tabular}{p{2in} p{5.5in}}
%                \begin{align*}
                        & \\
                        Alex Mullans & \line(1,0){285} \\ & \\
                        Ruben Rodriguez & \line(1,0){285} \\ & \\
                        Ethan Veatch & \line(1,0){285} \\ & \\
                        Kurtis Zimmerman & \line(1,0){285}
                        \end{tabular}
                \end{table}
%                \end{align*}
        \end{center}
\vspace*{\fill}
\thispagestyle{empty}

\clearpage

\tableofcontents

\clearpage
        
\section{Executive Summary}
This document is the fourth in a series of milestone documents that will accompany the planning of the Siftables\index{Siftables} Emulator\index{emulator}. The Emulator project is an application that will allow developers of Sifteo applications to test the features of Sifteo Cubes --- miniature computers that interact and communicate when used in tandem --- in a virtual programming environment. There is currently an emulator from Sifteo, Inc. that comes as part of the \gls{SDK}\index{SDK}\glsadd{SDKa} for the Cubes. However, Singularity intends to come up with a more natural interface than the one currently provided in that application.\\\\
This milestone defines the standards for code in the Siftables\index{Siftables} Emulator\index{emulator} project as well as the manner in which change will be controlled. It also elaborates the test cases that sufficiently cover the system. Future milestones will present design and usability reports as the software stabilizes.

\section{Introduction}
Developers of applications for the \gls{Sifteo Cubes}\index{Sifteo} currently must test programs they create for the platform within the emulator provided by Sifteo. While this emulator covers all  the functionality of the Sifteo Cubes, it presents a user interface that Singularity Software believes could be more naturally implemented. As such, Singularity Software will provide, in the form of the Siftables Emulator\index{emulator}, a new software-based emulator\index{emulator} for the Sifteo Cubes that will allow developers to more naturally interact with the platform.\\\\
Milestone 4 relies on previous milestones as it defines a change control plan, coding standards, and test cases. It follows Milestones 2 and 3, which laid the foundation and elaborated the requirements of the Siftables Emulator specification based on the high-level design created in Milestone 1. Milestone 5 will elaborate on the upcoming \gls{usability study} comparing the proposed Siftables Emulator design to the Sifteo\index{Sifteo} emulator's design before presenting original and feedback-based interface designs for the Siftables\index{Siftables} Emulator\index{emulator}.

\section{Project Background}
The Siftables Emulator is being developed by Singularity Software as part of the Junior Project sequence of classes at Rose-Hulman Institute of Technology. When projects were solicited for the sequence, clients Tim Ekl and Eric Stokes (both Rose-Hulman alumni) submitted a request for an emulator\index{emulator} for Sifteo Cubes, a new platform intended for ``intelligent play." After Singularity was chosen for the project, we met with Mr. Ekl to determine the three primary features of the Emulator: a Workspace where 1-6 Cubes could mimic the manipulations possible with physical Cubes, an \gls{API} to program those virtual Cubes, and a set of example games designed to show off the first two features. Singularity's Emulator\index{emulator} is intended to build on the foundation of Sifteo, Inc.'s existing emulator by creating a more fluid and natural user interface.

\section{Usability Report}
\subsection{Process}
\subsection{Analysis}
\subsection{Findings}

\section{Interaction Architecture}

\section{Initial Interface Design}

\appendix
    \begin{landscape}
    \section{Features}
    \begin{table}[h!]
      \begin{tabular}{p{1.5in} | p{2.25in} | p{.75in} | p{.75in} | p{.75in} | p{2.25in}}
        \textbf{Feature} &
        \textbf{Description} &
        \textbf{Status} &
        \textbf{Priority} &
        \textbf{Risk} &
        %\textbf{Stability} &
        \textbf{Reason} \\ \hline
        %\textbf{Effort} \\ \hline

        Individual, virtual Sifteo Cube &
        A virtual representation of a single Sifteo cube &
        Approved &
        Critical &
        Low &
        %High &
        Replicates physical Sifteo Cube \\ \hline
        %Medium \\ \hline

        Buttons to manipulate each virtual Cube &
        Buttons on the virtual Cube will allow the user to flip and tilt it &
        Approved &
        Critical &
        Medium &
        %High &
        Replaces physical actions where said actions would be impractical with a mouse \\ \hline
        %Medium \\ \hline

        Workspace where multiple cubes can be emulated &
        Multiple cubes will be displayed on a workspace that replicates the free-form nature of physical Sifteo Cubes\index{Sifteo Cubes} &
        Approved &
        Critical &
        Low &
        %High &
        Replicates multiple Sifteo Cubes\index{Sifteo Cubes} in a natural, free-form environment \\ \hline
        %High \\ \hline

        Interactions between Cubes &
        The Cubes present on the workspace will communicate when they are neighbored &
        Approved &
        Critical &
        Low &
        %High &
        Cubes can simulate the interactions possible with physical Cubes \\ \hline
        %High \\ \hline

        Load programs into the Cubes &
        The user will load his own and example programs into the emulator’s\index{emulator} Cubes &
        Approved &
        Critical &
        Medium &
        %High &
        The ability to program programs for the emulator\index{emulator} is dependent on a common interface \\ \hline
        %High \\ \hline

        Snap Cubes to invisible grid &
        The Cubes will snap into an invisible grid when a button is clicked &
        Proposed &
        Useful &
        Medium &
        %High &
        Increases productivity by allowing a quick reset if the Cubes are in disarray \\ \hline
        %Low \\ \hline

        Zoom Workspace &
        The Workspace will zoom to the level of an individual Cube or the whole space &
        Proposed &
        Useful &
        Low &
        %High &
        Inspecting individual Cubes allows for precise checks of program \glspl{GUI}\index{GUI}\glsadd{GUIa} \\ \hline
        %Low \\ \hline

      \end{tabular}
    \end{table}
    \end{landscape}

    \clearpage
    
\clearpage
\addcontentsline{toc}{section}{Glossary}
\printglossaries
\clearpage

\addcontentsline{toc}{section}{References}
\section*{References}

        \begin{enumerate}
                \item{Sifteo Inc. Online: http://www.sifteo.com}
                \item{Tim Ekl.  Client Meeting. 4 10 October 2011 4:15 p.m.}
                \item{Milestone 1.  Singularity Software.  Online: https://github.com/alexmullans/Siftables-Emulator/blob/master/docs/pdfs/m1.pdf}
                \item{Milestone 2.  Singularity Software.  Online: https://github.com/alexmullans/Siftables-Emulator/blob/master/docs/pdfs/m2.pdf}
                \item{Milestone 3.  Singularity Software.  Online: https://github.com/alexmullans/Siftables-Emulator/blob/master/docs/pdfs/m3.pdf}
                \item{Guido van Rossum.  ``PEP 8 Style Guide for Python Code.'' \\Online: http://www.python.org/dev/peps/pep-0008/}
                \item{``C\# Coding Standards Document.'' \\Online: http://weblogs.asp.net/lhunt/pages/CSharp-Coding-Standards-document.aspx}

        \end{enumerate}

\clearpage

\addcontentsline{toc}{section}{Index}
\printindex

\end{document}
