\title{Singularity Software\\Milestone 2}
\date{\today}

\documentclass[12pt]{article}
\usepackage[a4paper]{geometry}
\usepackage{makeidx}
\usepackage[acronym]{glossaries}
\usepackage{lscape}
\usepackage{amsmath}
% \usepackage{hyperref} % Makes links from ToC

\geometry{top=1.0in, bottom=1.0in, left=1.0in, right=1.0in} % Sets the margins

\setlength{\parindent}{0pt} % Fixes the paragraph spacing problem

% This is all for formatting and making the Table of Contents according to 
% spec. Don't play with it.
\makeatletter
\renewcommand\l@section[2]{%
  \ifnum \c@tocdepth >\z@
    \addpenalty\@secpenalty
    \addvspace{1.0em \@plus\p@}%
    \setlength\@tempdima{1.5em}%
    \begingroup
      \parindent \z@ \rightskip \@pnumwidth
      \parfillskip -\@pnumwidth
      \leavevmode \bfseries
      \advance\leftskip\@tempdima
      \hskip -\leftskip
      #1\nobreak\ 
      \leaders\hbox{$\m@th\mkern \@dotsep mu\hbox{.}\mkern \@dotsep mu$}
     \hfil \nobreak\hb@xt@\@pnumwidth{\hss #2}\par
    \endgroup
  \fi}
\makeatother

\makeindex

% Construct the glossary here
% Use the template below, then where the word appears (in the case below, computer), replace computer with \gls{computer}
\makeglossaries

\newglossaryentry{Sifteo Cubes}
{
  name={Sifteo Cubes},
  description={are small machines capable of loading programs and interacting with one another as well as responding to predefined movements}
}

\newglossaryentry{Object-Oriented Programming}
{
  name={Object-Oriented Programming},
  description={is a programming paradigm using objects to design applications}
}

\newglossaryentry{Windows}
{
  name={Windows},
  description={is a series of operating systems developed by Microsoft}
}

\newglossaryentry{Mac}
{
  name={Mac},
  description={is a series of lines of personal computers developed by Apple}
}

\newglossaryentry{Linux}
{
  name={Linux},
  description={is a Unix-based operating system based on free and open source software}
}

\newglossaryentry{cross-platform support}
{
  name={cross-platform support},
  description={is an attribute given to software implemented and operable on multiple computer platforms}
}

\newacronym{API}{API}{\glsadd{API}{Application Programming Interface}}

\newglossaryentry{APIg}
{
  name={Application Programming Interface},
  description={is an interface implemented by a software program that enables it to interact with other software}
}


\newglossaryentry{open source}
{
  name={open source},
  description={is an attribute given to software for which the source code is freely available}
}

\newacronym{IDE}{IDE}{\glsadd{IDE}{Integrated Development Environment}}

\newglossaryentry{IDEa}
{
  name={Integrated Development Environment},
  description={is software that provides a comprehensive work environment for computer programmers and software developers}
}


\newacronym{GUI}{GUI}{\glsadd{GUI}{Graphical User Interface}}

\newglossaryentry{GUIa}
{
  name={Graphical User Interface},
  description={is a visual way of allowing the user to interace with a computer program}
}

\newglossaryentry{version control}
{
  name={version control},
  description={is the management of documents and programs for a project over many versions in a well-organized manner}
}

\newglossaryentry{issue tracking system}
{
  name={issue tracking system},
  description={is a piece of software used to maintain a list of issues as generated during a project}
}

\renewcommand*\arraystretch{1.5}

\begin{document}
\vspace*{\fill}
        \begin{center}
                \LARGE{Singularity Software} \\
                \LARGE{\textit{Milestone 2}} \\
                \vspace{.15in}
                \large{\today} \\
                \vspace{4in}
                By signing below, I approve the contents of the following document. \\
                \begin{table}[h]
                        \begin{tabular}{p{2in} p{5.5in}}
%                \begin{align*}
                        & \\
                        Alex Mullans & \line(1,0){285} \\ & \\
                        Ruben Rodriguez & \line(1,0){285} \\ & \\
                        Ethan Veatch & \line(1,0){285} \\ & \\
                        Kurtis Zimmerman & \line(1,0){285}
                        \end{tabular}
                \end{table}
%                \end{align*}
        \end{center}
\vspace*{\fill}
\thispagestyle{empty}

\clearpage

\tableofcontents

\clearpage
        
\section{Executive Summary}
This document is the second in a series of milestone documents that will accompany the planning of the Siftables\index{Siftables} Emulator\index{emulator}. The Emulator project is a first of its kind application that will allow developers of Sifteo applications to test the features of the cubes in a virtual programming environment. Currently, the only way to test apps developed for the Sifteo Cubes platform is with the physical cubes; this project will eliminate that need and serve as a demo for the possibilities of the Sifteo platform.\\\\
This milestone elaborates the functional features of the the Siftables\index{Siftables} Emulator\index{emulator} project. It first gives a brief background of the project before moving on to elaborate features with use cases and declarative statements, where appropriate. It also maps all covered use cases to their relevant features. Finally, it presents mockups of the Emulator's design for consideration and to solicit client feedback.  Future milestones will present plans for change control, coding standardization, and testing. Finally, design and usability reports will make up the core of milestones near the end of the quarter as the software stabilizes.


\section{Introduction}
Developers of applications for the \gls{Sifteo Cubes}\index{Sifteo} currently must test programs they create for the platform on the Cubes themselves.  With a full release of the Cubes and corresponding \gls{API}\index{API}\glsadd{APIg} still pending, developers unable to join the Sifteo Early Access program are left without a software-based interface within which to productively develop Sifteo programs. As such, Singularity Software will provide, in the form of the Siftables Emulator\index{emulator}, a software-based emulator\index{emulator} for the Sifteo Cubes that will allow any developer to try programming in the unqiue environment provided by the Cubes.\\\\
Milestone 2 lays the foundation of the Siftables Emulator specification based on the high-level design created in Milestone 1. It will be supplemented by the specification and prototypes in Milestone 3. Milestone 4 will rely on these early milestones as they define a change control plan and test cases, and Milestone 5 will elaborate the usability guidelines and interface design that implement the features and use cases described herein.


\section{Project Background}
The Siftables Emulator is being developed by Singularity Software as part of the Junior Project sequence of classes at Rose-Hulman Institute of Technology. When projects were solicited for the sequence, clients Tim Ekl and Eric Stokes (both Rose-Hulman alumni) submitted a request for an emulator for Sifteo Cubes, a new platform intended for "intelligent play." After Singularity was chosen for the project, we met with Mr. Ekl to determine the three primary features of the Emulator: a Workspace where 1-6 cubes could mimic the manipulations possible with physical Cubes, an \gls{API} to program those virtual cubes, and a set of example games designed to show off the first two features. Singularity's Emulator will be the first program of its kind on the market for Sifteo Cubes.

\clearpage

\section{Use Cases}

  \subsection{Load program}

    \begin{description}
      \item[Name:] Load program
      \item[Description:] User selects program file and loads it into the emulator.
      \item[Actors:] User
      \item[Basic flow:] \hfill 
        \begin{enumerate}
	  \item{User presses "Load a program" button.}
	  \item{User selects *.siftem file in file dialog.}
	  \item{User presses "Open" button.}
	  \item{Emulator loads program on cubes in emulator.}
        \end{enumerate}
      \item[Alternate flows:] \hfill \\
	When the user opens an incompatible file (i.e. any file without the .siftem extension), or \\
	When the user opens a corrupt or otherwise unloadable file, or \\
	When the user presses "Cancel" button:
        \begin{enumerate}
	  \item{An error dialog summarizing the issue is presented to the user.}
	  \item{The use case terminates and no program is loaded.}
        \end{enumerate}
      \item[Pre-conditions:] \hfill
        \begin{enumerate}
          \item{Emulator is running.}
        \end{enumerate}
      \item[Post-conditions:] \hfill
        \begin{enumerate}
	  \item{Program is loaded or user cancelled loading.}
        \end{enumerate}
      \item[Special requirements] \hfill
        \begin{enumerate}
          \item{Emulator should indicate when program is finished loading.}
        \end{enumerate}
    \end{description}

  \subsection{Reload program}

    \begin{description}
      \item[Name:] Reload program
      \item[Description:] User reloads the current program file in the emulator.
      \item[Actors:] User
      \item[Basic flow:] \hfill
        \begin{enumerate}
	  \item{User presses "Reload this program" button.}
	  \item{User presses "Yes" on caution dialog.}
	  \item{Emulator loads program on cubes in emulator.}
        \end{enumerate}
      \item[Alternate flows:] \hfill \\
	When User presses "no" on caution dialog:
        \begin{enumerate}
          \item{The use case terminates and the program is not reloaded.}
        \end{enumerate}
      \item[Pre-conditions:] \hfill
        \begin{enumerate}
	  \item{A program is loaded in the emulator.}
        \end{enumerate}
      \item[Post-conditions:] \hfill
        \begin{enumerate}
	  \item{Program is loaded.}
        \end{enumerate}
      \item[Special requirements:] \hfill
        \begin{enumerate}
	  \item{Emulator should indicate when program is finished loading.}
        \end{enumerate}
    \end{description}

  \subsection{Zoom screen}

    \begin{description}
      \item[Name:] Zoom screen
      \item[Description:] User selects program file and loads it into the emulator.
      \item[Actors:] User
      \item[Basic flow:] \hfill
        \begin{enumerate}
	  \item{User adjusts zoom slider.}
	  \item{Emulator magnifies cubes in emulator according to zoom level.}
        \end{enumerate}
      \item[Alternate flows:] \hfill \\
	None
      \item[Pre-conditions:] \hfill
        \begin{enumerate}
          \item{Emulator is running.}
        \end{enumerate}
      \item[Post-conditions:] \hfill
        \begin{enumerate}
	  \item{Program running at beginning of use case, if any, is still running.}
        \end{enumerate}
    \end{description}	

  \subsection{Add/remove cubes}

    \begin{description}
      \item[Name:] Add/remove cubes
      \item[Description:] User adjusts the number of cubes present in the emulator.
      \item[Actors:] User
      \item[Basic flow:] \hfill
        \begin{enumerate}
	  \item{User adjusts "Number of cubes" slider or spinbox.}
	  \item{Emulator adds/removes cubes in emulator.}
        \end{enumerate}
      \item[Alternate flows:] \hfill \\
	None
      \item[Pre-conditions:] \hfill
        \begin{enumerate}
	  \item{Emulator is running.}
        \end{enumerate}
      \item[Post-conditions:] \hfill
        \begin{enumerate}
	  \item{Number of cubes has been adjusted to number specified.}
	  \item{Program running at beginning of use case, if any, is still running, but only on the cubes not newly added.}
        \end{enumerate}
    \end{description}

  \subsection{Snap cubes to grid}

    \begin{description}
      \item[Name:] Snap to grid
      \item[Description:] User resets the cubes to their initial positions in the emulator.
      \item[Actors:] User
      \item[Basic flow:] \hfill
        \begin{enumerate}
	  \item{User presses "Snap to Grid" button.}
	  \item{Emulator moves cubes to a grid orientation based on their current positions.}
        \end{enumerate}
      \item[Alternate flows:] \hfill \\
	None	
      \item[Pre-conditions:] \hfill
        \begin{enumerate}
	  \item{Emulator is running.}
        \end{enumerate}
      \item[Post-conditions:] \hfill
        \begin{enumerate}
	  \item{Cubes are arranged in a grid.}
        \end{enumerate}
    \end{description}

  \subsection{Manipulate cube}

    \begin{description}
      \item[Name:] Manipulate cube
      \item[Description:] User manipulates a cube by clicking buttons or the cube itself.
      \item[Actors:] User
      \item[Basic flow:] \hfill
        \begin{enumerate}
	  \item{User double clicks on cube.}
	  \item{Cube responds as if a screen click occured.}
        \end{enumerate}
      \item[Alternate flows:] \hfill
        \begin{enumerate}
	  \item{User clicks on buttons superimposed on cube edges.}
	  \item{Cube executes appropriate action (i.e. flips, rotates, tilts).}
        \end{enumerate}
        \begin{enumerate}
          \item{User drags cube next to another cube.}
	  \item{Cube communicates ("neighbors") with the cube(s) it is adjacent to.}
        \end{enumerate}
      \item[Pre-conditions:] \hfill
        \begin{enumerate}
          \item{Emulator is running.}
        \end{enumerate}
      \item[Post-conditions:] \hfill
        \begin{enumerate}
	  \item{If a program is running, the emulator has updated its state based on the cube's change.}
        \end{enumerate}
      \item[Special requirements:] ???
    \end{description}

  \subsection{API}
    The emulator will include an API in order to define a set of rules and specifications for the software.

  \subsection{Example games}
    The emulator will include games as examples for the user to demonstrate how the cubes interact with each other.

\section{Use Case Feature Mapping}


\section{Storyboards}

    \begin{landscape}
    \begin{table}[h]
      \begin{tabular}{p{.5in} | p{2.25in} | p{2.75in} | p{3in}}
        \textbf{ID} &
        \textbf{Feature} &
        \textbf{Description} &
        %\textbf{Status} &
        %\textbf{Priority} &
        %\textbf{Risk} &
        %\textbf{Stability} &
        \textbf{Reason} 
        %\textbf{Effort}
        \\ \hline

        F1 &
        Individual, virtual Sifteo Cube &
        A virtual representation of a single Sifteo cube &
        %Approved &
        %Critical &
        %Low &
        %High &
        Replicates physical Sifteo Cube
        %Medium 
        \\ \hline

        F2 &
        Buttons to manipulate each virtual Cube &
        Buttons on the virtual Cube will allow the user to flip and tilt it &
        %Approved &
        %Critical &
        %Medium &
        %High &
        Replaces physical actions where said actions would be impractical with a mouse
        %Medium
        \\ \hline

        F3 &
        Workspace where multiple cubes can be emulated &
        Multiple cubes will be displayed on a workspace that replicates the free-form nature of physical Sifteo Cubes\index{Sifteo Cubes} &
        %Approved &
        %Critical &
        %Low &
        %High &
        Replicates multiple Sifteo Cubes\index{Sifteo Cubes} in a natural, free-form environment
        %High
        \\ \hline

        F4 &
        Interactions between Cubes &
        The Cubes present on the workspace will communicate when they are neighbored &
        %Approved &
        %Critical &
        %Low &
        %High &
        Cubes can simulate the interactions possible with physical Cubes
        %High 
        \\ \hline

        F5 &
        Load programs into the Cubes &
        The user will load his own and example programs into the emulator’s\index{emulator} Cubes &
        %Approved &
        %Critical &
        %Medium &
        %High &
        The ability to program programs for the emulator\index{emulator} is dependent on a common interface
        %High
        \\ \hline

        F6 &
        Snap Cubes to invisible grid &
        The Cubes will snap into an invisible grid when a button is clicked &
        %Proposed &
        %Useful &
        %Medium &
        %High &
        Increases productivity by allowing a quick reset if the Cubes are in disarray
        %Low
        \\ \hline

        F7 &
        Zoom Workspace &
        The Workspace will zoom to the level of an individual Cube or the whole space &
        %Proposed &
        %Useful &
        %Low &
        %High &
        Inspecting individual Cubes allows for precise checks of program \glspl{GUI}\index{GUI}\glsadd{GUIa}
        %Low
        \\ \hline

      \end{tabular}
    \end{table}
    \end{landscape}


\clearpage
\addcontentsline{toc}{section}{Glossary}
\printglossaries
\clearpage

\addcontentsline{toc}{section}{References}
\section*{References}

        \begin{enumerate}
                \item{Sifteo Inc. Online: http://www.sifteo.com}
                \item{Tim Ekl.  Client Meeting.  12 September 2011 12:45 p.m.}
        \end{enumerate}

\clearpage

\addcontentsline{toc}{section}{Index}
\printindex

\end{document}
